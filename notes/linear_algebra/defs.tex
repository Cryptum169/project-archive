\subsubsection{Hermitian Matrix}
\paragraph{Introduction} Otherwise known as a self-adjoint matrix,
a Hermitian matrix is a \emph{complex square} matrix that is equal
to its own \emph{conjugate transpose}. In otherwords

\begin{equation*}
    A \text{ Hermitian } \Leftrightarrow A = \overline{{A^T}}
\end{equation*}

Concisely written as $A = A^H$

\paragraph{Properties}

Entries on the main diagonal of any hermitian matrix are real

\textcolor{red}{forgot why I included this definition, need investigation}

\textcolor{blue}{Perhaps the definition of a more general ``Symmetric matrix''}

\subsubsection{Definiteness}

\paragraph{Introduction} Symmetri $n \times n$ real matrix is said to be
positive-definite if scalar $z^TMz$ is strictly positive for every non-zero
column vector $z \in \mathcal{R}^n$. Semi-definite is defined similarly,
except above scalar must be non-negative. 

\paragraph{Properties} Squre root and Cholesky decomposition might be 
two relevant properties here.

A matrix is positive semidefinite iff. there is a positive semidefinite
matrix $B$ satisfying $M = BB$. $B$ is unique, and is called the 
non-negative square root of $M$, denoted with $B = M^{\frac{1}{2}}$.

$B$ is hermitian, so $B* = B$, complex conjugate of itself. 

\textcolor{blue}{Some use square root and $\sqrt{M}$ for any such dcomposition,
or specifically for the Cholesky decomposition, or any decomposition, other
only use for the non-negative square root.}

Related to Cholesky decomposition as $M = LL*$, where $L$ is lower
triangular with non-negative diagonal. See more for section
\ref{sec:cho-decomp} on Cholesky decomposition.

